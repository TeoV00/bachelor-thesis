% ! TeX root = ../../main.tex
\chapter{Introduzione}

%%------------ istruzioni -------%%
% il  quadro  di  riferimento,  che  descrive  il  contesto  in  cui  si  colloca  il  lavoro  di  tesi  e l’importanza  che  ha  questo  contesto  nell’attuale  stato  della  ricerca  o  delle  tecnologie (circa una pagina).

% L’obiettivo della tesi, che descrive gli scopi della tesi in riferimento al quadro introdotto (circa mezza pagina). 
% Al termine del tirocinio lo studente acquisirà competenze nell’ambito della progettazione e sviluppo di applicazioni mobile ibride (per es. Flutter). Inoltre, lo studente acquisirà competenze nell’ambito dell’utilizzo della realtà aumentata e della gamification per aumentare la consapevolezza dell’utente.

% Lo scopo di questo tirocinio e quello di sviluppare una applicazione mobile (ibrida) di realta aumentata con aspetti di gamification per fornire informazioni sui progressi raggiunti nel processo di dematerializzazione e pianutmazione della Universita di Bologna.
% I  risultati  raggiunti,  riassunti  a  partire  dalla  struttura  generale  della  tesi  (circa  una pagina).  Questa  parte  racconta  brevemente  i  contenuti  della  tesi  evidenziando  i contributi originali apportati dallo studente e offrendo  una visione organica del lavoro.

\section*{Sommario}
% Un sommario (lungo circa una pagina), organizzato per abstract dei singoli capitoli cheninizi  con  una  frase  introduttiva  del  tipo  “Il  seguito  della  tesi  è  così  organizzato”  e introduca  ad  uno  ad  uno  i  capitoli  ed  il  loro  contenuto  (“Il  capitolo  1  introduce  i linguaggi di markup per il Web e in  particolare tratta ….”).
Di seguito si fornisce una breve descrizione dell'organizzazione di questa tesi che presenta 3 capitoli: 
\begin{itemize}
    \item \textbf{Background e contesto} che introduce il contesto in cui si colloca l'elaborato di questa tesi e le tecnologie, e tecniche, utilizzate durante lo sviluppo andando, in prima battuta, a chiarire i significati dei termini \textit{sostenibilità} ed \textit{ecosostenibilità} proseguendo con una breve descrizione del progetto intrapreso dalle Nazioni Unite chiamato Agenda 2030 che prevede 17 Obiettivi per lo Sviluppo Sostenibile arrivando a presentare il progetto Re.Made, iniziativa dell'università di Bologna volta a rendere l'Università più ecosostenibile.
    Successivamente vengono discusse le tecniche utilizzate per motivare gli utenti, come la gamification e i serious game, in contesti non prettamente di gioco, vengono descritte le diverse tipologie di realtà estesa concludendo con l'analisi di alcuni dei principali strumenti e tecnologie utilizzate per l'integrazione di dispositivi di diversa natura soffermandosi anche sul caso specifico dello smartphone con il totem interattivo.
    \item \textbf{Design e tecnologie}, dove si entra nel vivo dell'elaborato, presenta l'applicazione BoschettoAR che è stata integrata con il totem, vengono analizzati e definiti i requisiti che l'applicativo del totem e l'app devono soddisfare al termine dello sviluppo, si passa poi alla descrizione delle scelte di design architetturale e grafico del totem e delle schermate dell'app mostrando schemi e mockup. Infine vengono elencate e brevemente descritte le tecnologie e gli strumenti di sviluppo utilizzati.
    \item \textbf{Implementazione}, terzo ed ultimo capitolo, in cui vengono illustrate le possibili alternative tecnologiche per l'integrazione dell'app mobile con il totem indicandone pregi e difetti per ciascuna, proseguendo viene mostrata l'organizzazione generale dei file del progetto per poi passare alla trattazione dell'implementazione delle funzionalità da inserire nell'app BoschettoAR mostrandone le schermate concluse e il codice particolarmente rilevante per la spiegazione di animazioni e design pattern utilizzati. Continuando con la lettura si arriva alla discussione dell'implementazione del totem e di tutte le sue schermate con particolare attenzione, anche in questo caso, a descrivere il codice che mette in evidenza l'uso di particolari pattern, la logica di business e le animazioni dell'interfaccia utente.
\end{itemize}