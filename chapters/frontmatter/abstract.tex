% ! TeX root = ../../main.tex
\chapter{Introduzione}

%%------------ istruzioni -------%%
% il  quadro  di  riferimento,  che  descrive  il  contesto  in  cui  si  colloca  il  lavoro  di  tesi
Sempre di più, al giorno d'oggi, si parla di clima, sostenibilità ed ecosostenibilità, visti anche i recenti avvenimenti climatici alquanto estremi che evidenziano una grandissima alterazione degli equilibri degli ecosistemi del pianeta terra e del mondo animale dovuti, totalmente o in parte, dalle attività umane.
\\

Con lo scopo di colmare la poca sensibilità e l'assenza d'interesse sulla salute della natura e sui temi sociali avuti fino al 2015, l'organizzazione mondiale delle Nazioni Unite ha intrapreso diversi percorsi ed iniziative per il raggiungimento di obiettivi volti a risolvere i problemi che colpiscono il mondo; anche l'università di Bologna ha preso parte al cambiamento attraverso progetti e iniziative che prevedono il raggiungimento di alcuni obiettivi definiti dall'ONU.
\\

Esistono molteplici modi per diffondere informazione, consapevolezza ed educare e fra questi rientrano anche tecnologie di \textit{Extended Reality} che, a un primo sguardo, principalmente vengono utilizzate in sistemi d'intrattenimento video ludici come i videogiochi ma che possono essere sfruttate anche in contesti di natura ben diversa portando a diversi vantaggi nel mondo educativo e professionale.

Un altro strumento che può essere utilizzato, specialmente per migliorare e aumentare il coinvolgimento e l'interesse delle persone, è la \textit{gamification}, una tecnica che prevede di strutturare attività quotidiane e noiose come fossero videogiochi presentando obiettivi, livelli e premi, oltre a introdurre dinamiche presenti nei giochi.
\\

Il progetto presentato in questa tesi si pone l'obiettivo di sviluppare un sistema che prevede l'utilizzo di dispositivi di diversa natura, come lo smartphone e il totem interattivo, per la fruizione di un'esperienza in realtà aumentata che porti ad aumentare la consapevolezza e la conoscenza sui progressi raggiunti dall'Università di Bologna per quanto concerne la sostenibilità ambientale e più precisamente la piantumazione di alberi e la riduzione dell'uso di carta dei processi amministrativi e comunicativi.
Nel dettaglio verrà sviluppato un software per un totem interattivo che, integrato con l'app BoschettoAR precedentemente sviluppata, mostrerà un bosco virtuale, una classifica degli utenti e i progressi da loro raggiunti. Nello sviluppo è previsto l'utilizzo di tecnologie per lo sviluppo di applicazioni multi piattaforma, di servizi appartenenti al \textit{cloud computing} e di un sistema d'identificazione basato sui codici QR.

\section*{Sommario}
Di seguito si fornisce una breve descrizione dell'organizzazione di questa tesi che presenta 3 capitoli: 
\begin{itemize}
    \itemsep1em
    \item \textbf{Background e contesto} che introduce il contesto in cui si colloca l'elaborato di questa tesi e le tecnologie, e tecniche, utilizzate durante lo sviluppo andando, in prima battuta, a chiarire i significati dei termini \textit{sostenibilità} ed \textit{ecosostenibilità} proseguendo con una breve descrizione del progetto intrapreso dalle Nazioni Unite chiamato Agenda 2030 che prevede 17 Obiettivi per lo Sviluppo Sostenibile arrivando a presentare il progetto Re.Made, iniziativa dell'università di Bologna volta a rendere l'Università più ecosostenibile.
    Successivamente vengono discusse le tecniche utilizzate per motivare gli utenti, come la gamification e i serious game, in contesti non prettamente di gioco, vengono descritte le diverse tipologie di realtà estesa concludendo con l'analisi di alcuni dei principali strumenti e tecnologie utilizzate per l'integrazione di dispositivi di diversa natura soffermandosi anche sul caso specifico dello smartphone con il totem interattivo.
    \item \textbf{Design e tecnologie}, dove si entra nel vivo dell'elaborato, presenta l'applicazione BoschettoAR che è stata integrata con il totem, vengono analizzati e definiti i requisiti che l'applicativo del totem e l'app devono soddisfare al termine dello sviluppo, si passa poi alla descrizione delle scelte di design architetturale e grafico del totem e delle schermate dell'app mostrando schemi e mockup. Infine vengono elencate e brevemente descritte le tecnologie e gli strumenti di sviluppo utilizzati.
    \item \textbf{Implementazione}, terzo ed ultimo capitolo, in cui vengono illustrate le possibili alternative tecnologiche per l'integrazione dell'app mobile con il totem indicandone pregi e difetti per ciascuna, proseguendo viene mostrata l'organizzazione generale dei file del progetto per poi passare alla trattazione dell'implementazione delle funzionalità da inserire nell'app BoschettoAR mostrandone le schermate concluse e il codice particolarmente rilevante per la spiegazione di animazioni e design pattern utilizzati. Continuando con la lettura si arriva alla discussione dell'implementazione del totem e di tutte le sue schermate con particolare attenzione, anche in questo caso, a descrivere il codice che mette in evidenza l'uso di particolari pattern, la logica di business e le animazioni dell'interfaccia utente.
\end{itemize}