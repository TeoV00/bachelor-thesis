% ! TeX root = ../../main.tex
\chapter{Introduzione}

%%------------ istruzioni -------%%
% il  quadro  di  riferimento,  che  descrive  il  contesto  in  cui  si  colloca  il  lavoro  di  tesi
Ad oggi, si parla sempre più di clima, sostenibilità ed ecosostenibilità. I recenti avvenimenti climatici alquanto estremi evidenziano una grandissima alterazione degli equilibri degli ecosistemi del pianeta Terra e del mondo animale dovuti, totalmente o in parte, alle attività umane.
\vspace{10pt}

Con lo scopo di colmare la poca sensibilità e l'assenza d'interesse sulla salute della natura e sui temi sociali avuti fino al 2015, l'organizzazione mondiale delle Nazioni Unite ha intrapreso diversi percorsi ed iniziative per il raggiungimento di obiettivi volti a risolvere i problemi che colpiscono il mondo. Anche l'università di Bologna ha preso parte a questo cambiamento attraverso progetti e iniziative che prevedono il raggiungimento di alcuni obiettivi definiti dall'ONU.
\vspace{10pt}

Esistono molteplici modi per diffondere educazione, informazione, e consapevolezza. Fra questi rientrano anche le tecnologie di \textit{Extended Reality} (ER). Le ER vengono principalmente  utilizzate in sistemi d'intrattenimento video-ludici come i videogiochi, ma che possono essere sfruttate anche in contesti di natura ben diversa portando a diversi vantaggi nel mondo educativo, medico e professionale.
\vspace{10pt}

Un altro strumento che può essere utilizzato, specialmente per migliorare e aumentare il coinvolgimento e l'interesse delle persone, è la \textit{gamification}. Si tratta di una tecnica che permette di strutturare attività quotidiane e noiose sotto forma di gioco presentando obiettivi, livelli e premi.
\vspace{10pt}

Il progetto presentato in questa tesi si pone l'obiettivo di sviluppare un sistema \textit{multi-device} tramite dispositivi come lo smartphone e il totem interattivo che aumenti la consapevolezza e la conoscenza sui progressi di sostenibilità ambientale raggiunti dall'Università di Bologna.
\vspace{10pt}

Più precisamente attraverso un'esperienza di Realtà Aumentata, è possibile venire a conoscenza degli obiettivi raggiunti dall'Ateneo come la piantumazione di alberi e la riduzione dell'uso di carta nei processi amministrativi e comunicativi.
Il totem interattivo, grazie ai dati ricevuti dall'applicazione BoschettoAR su smartphone, rappresenta un bosco virtuale, genera una classifica degli utenti più virtuosi e riporta i progressi da loro raggiunti cercando di sfruttare al meglio gli elementi della \textit{gamification}.
\vspace{10pt}

Nello sviluppo è previsto l'utilizzo di tecnologie per lo sviluppo di applicazioni multi piattaforma, di servizi appartenenti al \textit{cloud computing} e di un sistema d'identificazione basato sui codici QR oltre alla applicazione di tecniche di progettazione e di prototipazione.

\section*{Sommario}
Di seguito si fornisce una breve descrizione dell'organizzazione di questa tesi che presenta 3 capitoli: 
\begin{itemize}
    \itemsep1em
    \item \textbf{Background e contesto}. Viene introdotto lo scenario in cui si colloca l'elaborato di questa tesi e le tecnologie utilizzate durante lo sviluppo andando a chiarire i significati dei termini \textit{sostenibilità} ed \textit{ecosostenibilità}. Si prosegue con una breve descrizione del progetto intrapreso dalle Nazioni Unite, chiamato Agenda 2030, che prevede 17 Obiettivi per lo Sviluppo Sostenibile per poi presentare il progetto Re.Made, iniziativa dell'università di Bologna volta a rendere l'Università più ecosostenibile.
    Successivamente vengono discusse le tecniche utilizzate per motivare gli utenti, come la \textit{gamification} e i \textit{serious game}, in contesti non prettamente di gioco. Infine, vengono analizzate le diverse tipologie di Realtà Estesa concludendo con l'analisi di alcuni dei principali strumenti e tecnologie utilizzate per l'integrazione dei dispositivi, approfondendo il caso specifico dello smartphone con il totem interattivo.
    \item \textbf{Design e tecnologie}. Qui si entra nel vivo dell'elaborato: viene presentata l'applicazione BoschettoAR, integrata con il totem e vengono analizzati e definiti i requisiti che l'applicativo del totem e l'app devono soddisfare al termine dello sviluppo. Si passa poi alla descrizione delle scelte di design architetturale e grafico del totem e delle schermate dell'app mostrando schemi e mockup. Infine vengono elencate e brevemente descritte le tecnologie e gli strumenti di sviluppo utilizzati.
    \item \textbf{Implementazione}. Ultimo capitolo in cui vengono illustrate le possibili alternative tecnologiche per l'integrazione dell'app mobile con il totem, indicandone pregi e difetti per ciascuna. Successivamente viene mostrata l'organizzazione generale dei file del progetto per poi passare alla trattazione dell'implementazione delle funzionalità dell'app BoschettoAR e del totem.
    Per entrambi viene discusso lo sviluppo delle funzionalità con l'aiuto di porzioni di codice, di schemi UML e d'immagini che mostrano l'interfaccia utente (UI) finale.
\end{itemize}