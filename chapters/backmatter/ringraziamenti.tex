\chapter{Ringraziamenti}
\thispagestyle{empty}

Finalmente sono arrivato alla fine di questo percorso, fatto di esami, divertimento e tanti progetti di gruppo!!

In primis ringrazio la relatrice Dott.ssa Catia Prandi e la co-relatrice Dott.ssa Chiara Ceccarini che mi hanno seguito e spronato nello sviluppo dell'elaborato e nella stesura di questa tesi.
\vspace{10pt}

Ringrazio la mia Famiglia (nessuno escluso) che mi ama incondizionatamente. Grazie ai miei genitori, Ilva e Francesco, che hanno sempre fatto il possibile per permettermi di studiare ed avere la libertà di seguire il percorso che più mi piaceva, senza mai fare pressioni nella scelta o per gli esami, ma anzi rasserenandomi. 

Un grazie particolare a mia sorella Michi per avermi aiutato a correggere la tesi nonostante fosse già impegnata per i suoi esami e la sorellona Manu che da Milano mi faceva innervosire per gli abbonamenti scaduti non pagati.
\vspace{10pt}

Grazie ai miei amici d'infanzia Lucio e Filippo e alle bevute fatte assieme che mi hanno fatto passare momenti di spensieratezza.
\vspace{10pt}

Ringrazio la mia migliore amica, ormai sorella acquisita, Elvira con la quale ho passato momenti di festa e felicità, di tristezza condivisa, di Netflix, di condivisione dello stress pre-esami e tanta tanta allegria, carica e divertimento nelle feste a Bologna per ballare senza pensieri e giudizi.
\vspace{10pt}

Ringrazio il mio Amore Luca, che mi ha dato molteplici aiuti durante la stesura delle tesi e negli esami, mi ha sostenuto e mi sostiene in ciò che faccio dandomi l'energia per non demoralizzarmi davanti le difficoltà, come un faro durante la tempesta. Ma non è finita qui mio caro! Caricati di tanta pazienza che ne avrai bisogno per sopportarmi quando inizierò la magistrale <3
\vspace{10pt}

Ultimo ma non meno importante, ringrazio me stesso per essere stato meno autocritico ed aver creduto in me stesso, per essere riuscito, non so come, a dare tutti gli esami ed uscirne sano di mente anche se un po' contratto. Sii sempre \textit{PROUD} in quello che fai!!
\vspace{10pt}

Umh... dimentico qualcuno? Maccerto, Luana! La ragazza più romana che conosca, fissata per i gradienti e il gicoo taboo, con la quale ho avuto l'onore di ridere, scherzare, parlare, cantare anche, pranzare, bere, ballare e lavorare in gruppo durante questi tre anni intensi.
\vspace{20pt}

Desidero ringraziare anche la regione Emilia-Romagna che, attraverso l'ente ER.GO, fornisce interventi e servizi agli studenti universitari. Grazie perché con le vostre borse di studio aiutate tanti ragazzi e famiglie a sostenere le spese connesse all’università (es. libri) e non solo.