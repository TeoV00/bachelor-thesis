\chapter{Conclusioni}
%%------------- PARTE RIASSUNTIVA TESI -------------%%
% Nelle conclusioni, invece, si fa una prima parte molto riassuntiva del volume di tesi, per poi raccontare gli sviluppi futuri (di solito è più corto dell’introduzione). 
In questo elaborato si è descritto lo sviluppo dell'applicativo del totem e della sua integrazione con l'app mobile BoshcettoAR iniziando dalla fase di design architetturale e grafico, con l'aiuto di strumenti come schemi e mockup, fino ad arrivare alla fase d'implementazione in cui è stato spiegato nel dettaglio lo sviluppo del totem e delle nuove schermate dell'app mostrandone il codice e gli screenshot del risultato finale ottenuto.

Nella progettazione sono stati utilizzati \textit{game elements} come i badge, i livelli e la classifica in modo da giovare dei vantaggi che porta la gamification, tecnologie come il cloud computing per il salvataggio dei dati, i QR code per l'identificazione del totem e degli alberi e, infine, la Realtà Aumentata per migliorare la comprensione dei dati mostrati attraverso l'utilizzo di oggetti virtuali come, ad esempio, plichi di carta e taniche di benzina.

In particolare integrando l'app BoschettoAR con il totem interattivo si è cercato di sensibilizzare ulteriormente gli utenti sul tema della dematerializzazione e della piantumazione di alberi e sul come sia possibile raggiungere grandi risultati ecologici, come la riduzione di CO\textsubscript{2}, attraverso la partecipazione e collaborazione e di tutti.

Grazie al totem, gli utenti sono in grado di tenere traccia dei loro progressi confrontandoli con quelli degli altri attraverso la classifica e la visualizzazione dei progressi raggiunti da ogni singolo utente, in una sorta di competizione, e di contribuire nella creazione di un bosco virtuale sempre più ricco e verde aumentando la consapevolezza della comunità universitaria.

Benché il sistema sia stato testato su un solo totem, l’architettura è stata progettata in un’ottica di anticipazione dei cambiamenti, affinché possa essere facilmente scalabile e possano essere aggiunti più totem nelle diverse sedi dell’Ateneo.

%%------------- SVILUPPI FUTURI -------------%%
\section*{Possibili sviluppi futuri}
Nonostante siano stati raggiunti tutti gli obiettivi prefissati, definiti nei requisiti presentati nel secondo capitolo, è presente ancora un buon margine di miglioramento e ampliamento con nuove funzionalità.
\begin{itemize}
    \itemsep1em
    \item Per poter migliorare e incrementare il coinvolgimento e l'esperienza utente occorrerebbe implementare una dinamica più stimolante dello sblocco dei badge e del passaggio al livello successivo, proponendo sfide o quiz periodici da superare, anche attraverso l'uso della Realtà Aumentata.
    \item Sarebbe inoltre molto interessante diffondere questo progetto nelle altre sedi dell'Ateneo introducendo sfide ed eventi che le vedano gareggiare con la possibilità d'inserire anche nei totem il concetto dei badge o degli obiettivi.
    \item Per impedire che gli utenti perdano i propri dati e progressi dell'app mobile si potrebbe offrire la possibilità di esportare i dati in un file o di permettere il salvataggio sul cloud il che porterebbe all'introduzione della creazione facoltativa di un account utente.
    \item La sessione di Realtà Aumentata potrebbe essere aggiornata inserendo nuovi oggetti da disporre nell'ambiente circostante in modo da offrire agli utenti ulteriori termini di paragone dei risparmi.
\end{itemize}
