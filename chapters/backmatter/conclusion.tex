\chapter{Conclusioni}
%%------------- PARTE RIASSUNTIVA TESI -------------%%
Con questo volume di tesi si è voluto presentare e sviluppare, con l'aiuto della tecnologia AR e le tecniche di gamification, un nuovo modo per sensibilizzare le persone su temi come la sostenibilità ed ecosostenibilità, molto spesso percepiti lontani dalla realtà di ognuno. In particolare si è parlato di dematerializzazione e di come, insieme alla piantumazione di alberi, sia possibile risparmiare una grande quantità di materie come carta, elettricità, benzina riducendo l'emissione di CO\textsubscript{2}; il tutto fruibile attraverso l'app mobile BoschettoAR che ha permesso di mostrare virtualmente nel mondo reale, attraverso uno schermo, la mole di carta e di benzina risparmiate con questi processi.

Volendo rendere questa esperienza più avvincente e meno statica si è introdotto il totem interattivo, un device per creare un punto d'incontro fra i singoli utenti creando, attraverso tecniche di gamification, una competizione volta ad accrescere la consapevolezza di comunità sui temi trattati in questo elaborato.

L'utilizzo di tecnologie come la Realtà Estesa e tecniche di gamification si rivelano, nella maggior parte dei casi, ottimi strumenti per la diffusione di conoscenza e istruzione.

%%------------- SVILUPPI FUTURI -------------%%

Nonostante siano stati raggiunti gli obiettivi preposti, elencati nel capitolo di design, c'è un ampio margine di miglioramento e integrazione con altre applicazioni e sistemi gamificati sviluppati all'interno del progetto Re.Made.
Il sistema di gamificazione sia nell'app che nell'applicativo sviluppato per il totem può essere integrato e migliorato rendendolo più ricco di contenuti soffermandosi con più cura all'aspetto dei serious-game.

Per il momento il totem mostra la classifica degli utenti che hanno condiviso i dati con il totem di una specifica sede ma potrebbe essere molto interessante introdurre anche una classifica fra le sedi dell'Ateneo

Nelle conclusioni, invece, si fa una prima parte molto riassuntiva del volume di tesi, per poi raccontare gli sviluppi futuri (di solito è più corto dell’introduzione). 