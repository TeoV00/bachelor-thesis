\chapter{Conclusioni}
%%------------- PARTE RIASSUNTIVA TESI -------------%%

In questo elaborato si è descritto lo sviluppo dell'applicativo del totem e della sua integrazione con l'app mobile, utilizzando diverse tecnologie, partendo dalla fase di design architetturale e grafico, con l'aiuto di strumenti come schemi e mockup, fino ad arrivare alla fase d'implementazione in cui vengono spiegate nel dettaglio le implementazioni delle schermate del totem mostrandone il codice e gli screenshot del risultato finale.

Nella progettazione sono stati utilizzati elementi di gioco come i badge, i livelli e la classifica in modo da giovare dei vantaggi che porta la gamification e di tecnologie come il cloud computing per il salvataggio dei dati, i QR code e la Realtà Aumentata che permette di concretizzare, agli occhi degli utenti, i dati numerici che vengono mostrati sotto forma di oggetti virtuali (plichi di carta e taniche di benzina), migliorandone la comprensione.

In particolare integrando l'app BoschettoAR con il totem interattivo si è cercato di sensibilizzare ulteriormente gli utenti sul tema della dematerializzazione e della piantumazione di alberi e sul come sia possibile raggiungere grandi risultati ecologici, come la riduzione di CO\textsubscript{2}, attraverso la partecipazione e collaborazione e di tutti.

Grazie al totem, gli utenti sono in grado di tenere traccia dei loro progressi confrontandoli con quelli degli altri attraverso la classifica e la visualizzazione dei progressi raggiunti da ogni singolo utente, in una sorta di competizione, e di contribuire nella creazione di un bosco virtuale sempre più ricco e verde aumentando la consapevolezza della comunità universitaria.

%%------------- SVILUPPI FUTURI -------------%%
Nonostante siano stati raggiunti tutti gli obiettivi prefissati, definiti dai requisiti presentati nel secondo capitolo, è presente ancora un buon margine di miglioramento e ampliamento con nuove funzionalità.

Per poter migliorare e incrementare il coinvolgimento degli utenti occorrerebbe implementare una dinamica di sblocco più stimolante dei badge e del passaggio al livello successivo, proponendo sfide o quiz periodici da superare, anche attraverso l'uso della Realtà Aumentata.

Sarebbe inoltre molto interessante diffondere questo progetto nelle altre sedi dell'Ateneo introducendo sfide ed eventi che le vedano coinvolte con la possibilità d'inserire anche nei totem il concetto dei badge.

Per impedire che gli utenti perdano i propri dati e progressi dell'app mobile si potrebbe offrire la possibilità di esportare i dati in un file o di permettere il salvataggio sul cloud il che porterebbe all'introduzione della creazione facoltativa di un account dell'utente.

% Nelle conclusioni, invece, si fa una prima parte molto riassuntiva del volume di tesi, per poi raccontare gli sviluppi futuri (di solito è più corto dell’introduzione). 