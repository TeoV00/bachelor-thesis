\chapter{Conclusioni}
%%------------- PARTE RIASSUNTIVA TESI -------------%%
% Nelle conclusioni, invece, si fa una prima parte molto riassuntiva del volume di tesi, per poi raccontare gli sviluppi futuri (di solito è più corto dell’introduzione).
% \todo{manca una parte iniziale in cui rispieghi l’obiettivo della tesi e un attimo il contesto (dematerializzazione).}

Il processo della dematerializzazione continua a diffondersi e presentarsi nelle vite di persone e attività private o pubbliche.
Questo processo prevedere la crescente riduzione fino all'azzeramento dell'uso di carta passando a soluzioni totalmente digitali. Ad esempio, dal 2016 la raccolta delle opinioni anonime degli studenti sui corsi frequentati UniBo, viene svolta direttamente online evitando la stampa di migliaia e migliaia di fogli per i questionari.
\vspace{\baselineskip}

L'Università di Bologna in aggiunta alla diminuzione della carta ha effettuato, in modo proporzionale ai fogli risparmiati, la piantumazione di alberi ottenendo un risultato strabiliante in termini di ecosostenibilità. Questi progressi però, risultano difficili da immaginare considerate le elevate quantità di carta e di Kg di CO\textsubscript{2} risparmiate.
Vengono perciò in aiuto tecnologie e tecniche come la Realtà Mista, nello specifico la Realtà Aumentata (AR), e la gamification con i suoi \textit{game elements}. La prima permette di visualizzare nel mondo reale i fogli carta e l'anidride carbonica (convertita in benzina), mentre la seconda mette a disposizione gli strumenti per rendere divertente e stimolante l'esperienza generale di apprendimento.
\vspace{10pt}

L'obiettivo, che è stato raggiunto, di questa tesi era quello d'implementare un sistema multi-device che combinasse i pregi e le potenzialità dell'AR e della gamification per aumentare la consapevolezza sul tema della sostenibilità e la dematerializzazione in UniBo.
Più precisamente in questo elaborato si è descritto lo sviluppo dell'applicativo del totem e della sua integrazione con l'app mobile BoschettoAR. Ciò è avvenuto partendo con la fase di design architetturale e grafico, con l'aiuto di strumenti come schemi e mockup, fino ad arrivare alla fase d'implementazione in cui è stato spiegato nel dettaglio lo sviluppo del totem e delle nuove schermate dell'app. \`E stato quindi mostrato il codice e gli screenshot del risultato finale ottenuto.
\vspace{\baselineskip}
\\
Nella progettazione sono stati utilizzati \textit{game elements} (badge, livelli e classifiche) per giovare dei vantaggi della gamification; tecnologie come il cloud computing per il salvataggio dei dati; i QR code per l'identificazione del totem e degli alberi. Infine si è sfruttata la Realtà Aumentata per migliorare la comprensione dei dati, mostrati attraverso l'utilizzo di oggetti virtuali come, ad esempio, plichi di carta e taniche di benzina.

In particolare, integrando l'app BoschettoAR con il totem interattivo si è cercato di sensibilizzare ulteriormente gli utenti sul tema della dematerializzazione e della piantumazione di alberi e sul come sia possibile raggiungere grandi risultati ecologici, come la riduzione di CO\textsubscript{2}, attraverso la partecipazione e collaborazione di tutti.

Grazie al totem, gli utenti sono in grado di condividere i loro progressi raggiunti nell'app, di confrontarsi con gli altri attraverso la classifica in una sorta di competizione e di contribuire nella creazione di un bosco virtuale sempre più ricco e verde aumentando la consapevolezza della comunità universitaria.
\vspace{\baselineskip}

Benché il sistema sia stato testato su un solo totem, l’architettura è stata progettata in un’ottica di anticipazione dei cambiamenti, affinché possa essere facilmente scalabile e possano essere aggiunti più totem nelle diverse sedi dell’Ateneo.

%%------------- SVILUPPI FUTURI -------------%%
\vspace{\baselineskip}
Nonostante siano stati raggiunti tutti gli obiettivi prefissati, definiti nei requisiti presentati nel secondo capitolo, è presente ancora un buon margine di miglioramento e ampliamento con nuove funzionalità:
\begin{itemize}
    \itemsep1em
    \item Per poter migliorare e incrementare il coinvolgimento e l'esperienza utente, occorrerebbe impiegare dinamiche più stimolanti per l'ottenimento di nuovi badge e aumento di livello, proponendo sfide o quiz periodici da superare, anche attraverso l'uso della Realtà Aumentata;
    \item Sarebbe molto interessante diffondere questo progetto nelle altre sedi dell'Ateneo introducendo sfide ed eventi che le vedano gareggiare. In questo modo si presenterebbe la possibilità d'inserire anche nei totem elementi della gamification come ad esempio obiettivi, premi e livelli collettivi diversamente da quelli individuali presenti nell'app;
    \item Per garantire la sicurezza dei dati utente dell'app mobile, si potrebbe introdurre la funzionalità di esportazione dei dati, offrendo la possibilità di salvarli in un file locale o di sincronizzarli nel cloud. Quest'ultima opzione richiederebbe la creazione di un account in modo da poter gestire correttamente, e in sicurezza, le informazioni personali;
    \item La sessione di Realtà Aumentata potrebbe essere aggiornata inserendo nuovi oggetti (oltre a risme di carta e taniche di benzina) da disporre nell'ambiente circostante per offrire agli utenti ulteriori termini di paragone dei risparmi.
\end{itemize}
