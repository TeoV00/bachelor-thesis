% ! TeX root = ../../main.tex
\chapter{Design e tecnologie}
\section{Premessa}
\subsection{App Mobile}
L'applicazione mobile, da integrare con il totem, si chiama BoschettoAR. Questa applicazione, attraverso la realtà aumentata e la gamification, fornisce agli utenti informazioni sui progressi raggiunti dal'Università di Bologna attraverso il processo di dematerializzazione e piantumazione proporzionale prevista dal progetto ReMade (vedi sezione \ref{sec:remade}). Di seguito si descrive brevemente il funzionamento dell'app.

L'utente, trovato un albero appartenente a un bosco digitale all'interno di aree verdi dell'ateneo, apre l'applicazione, inquadra il codice QR sull'albero (figura \ref{fig:scanQRapp}) e una volta riconosciuto inizia l'esperienza di realtà aumentata che permette all'utente di comprendere più concretamente, e vedere con i propri occhi, l'ammontare di carta e di anidride carbonica risparmiata. In questa sessione di AR, l'utente ha la possibilità di posizionare sul terreno plichi di carta e taniche di benzina (figure \ref{fig:paperAR} e \ref{fig:tankAR}) in quantità proporzionale alla CO2 evitata dall'Università; vengono mostrate una breve descrizione dell'albero e dell'attività UniBo associata e anche informazioni sui risparmi ottenuti in termini di CO2, carta, elettricità e benzina.

\begin{figure}
    \centering
    \subfloat[Scansione QR code albero]{
        \includegraphics[width=0.30\textwidth]{img/app/scan_qr_old.jpg}
        \label{fig:scanQRapp}
    }
    \subfloat[Taniche di benzina in AR]{
        \includegraphics[width=0.30\textwidth]{img/app/ar_fuel.jpg}
        \label{fig:tankAR}
    }
    \subfloat[Plichi di carta in AR]{
        \includegraphics[width=0.30\textwidth]{img/app/ar_paper.jpg}
        \label{fig:paperAR}
    }
    \caption{Sessione AR dell'app mobile}
    \label{fig:arSession}
\end{figure}

Dopo la scansione di un albero, questo viene memorizzato nella collezione dell'utente che guadagna un badge aumentando di livello. Tutti i progressi raggiunti e i risparmi virtualmente guadagnati vengono visualizzati all'interno della pagina personale (figura \ref{fig:userPage}) dove è possibile anche impostare i dati del profilo e una foto.

\begin{figure}
    \centering
    \subfloat[Profilo utente con progressi e badge acquisiti]{
        \includegraphics[width=0.30\textwidth]{img/app/user_page.png}
        \label{fig:userProgress}
    }
    \subfloat[Pagina di modifica dei dati utente]{
        \includegraphics[width=0.30\textwidth]{img/app/edit_user.png}
        \label{fig:editUser}
    }
    \caption{Pagina utente app mobile}
    \label{fig:userPage}
\end{figure}

Gli alberi collezionati si trovano nella pagina principale (figura \ref{fig:treeHome}) con tutte le informazioni e la possibilità di avviare direttamente la sessione di realtà aumentata (figura \ref{fig:treeDetail}).
L'app è in grado di funzionare offline dopo una prima sincronizzazione necessaria.

\begin{figure}
    \centering
    \subfloat[Homepage alla prima installazione]{
        \includegraphics[width=0.30\textwidth]{img/app/first_launch.png}
        \label{fig:appFirstLaunch}
    }
    \subfloat[Collezione utente degli alberi scansionati]{
        \includegraphics[width=0.30\textwidth]{img/app/treeHome.png}
        \label{fig:treeHome}
    }
    \subfloat[Pagina dei dettagli dell'albero con possibilità di avviare l'AR]{
        \includegraphics[width=0.30\textwidth]{img/app/treeDetailsOld.png}
        \label{fig:treeDetail}
    }
    \caption[Homepage app BoschettoAR]{Homepage app BoschettoAR e pagina con dettagli albero}
    \label{fig:homeApp}
\end{figure}

\subsection{Obiettivo}
L'obiettivo di questa tesi è di estendere l'esperienza fornita dall'app BoschettoAR integrando un totem interattivo che mostri il livello di consapevolezza raggiunta dalla comunità UniBo sul tema dell'inquinamento attraverso la visualizzazione dei dati (es. bosco virtuale) grazie agli utenti che condividono i propri progressi tramite app ed aumentano la propria consapevolezza.

\section{Requisiti e analisi}
\subsection{Requisiti funzionali}
\subsubsection{App mobile}
\begin{itemize}
    \item Condivisone dei progressi: l'utente, dopo aver impostato un nickname, dovrà poter condividere i propri progressi raggiunti con il totem attraverso una schermata dedicata. Entro un certo intervallo di tempo il caricamento deve interrompersi automaticamente.
\end{itemize}
\subsubsection{Totem}
\begin{itemize}
    \item Devono essere presenti le schermate: homepage, statistiche, classifica e informazioni.
    \item Ricezione dei progressi: il totem deve ricevere i progressi dell'utente condivisi tramite app.
    \item La homepage deve mostrare gli utenti coinvolti e i dettagli dei loro progressi 
    \item La schermata delle statistiche deve mostrare il conteggio totale di alberi e progetti coinvolti, la carta, l'elettricità, la benzina, la corrente elettrica e la co2 virtualmente risparmiate con la partecipazione degli utenti. Si deve mostrare una breve descrizione per ciascun dato mostrato.
    \item La classifica deve mostrare i 10 migliori utenti sulla base della co2 risparmiata.
    \item Nella pagina delle informazioni deve essere presente la spiegazione dei calcoli effettuati e una sezione con maggiori dettagli sul progetto ReMade. Deve essere possibile aggiungere nuove sezioni.
\end{itemize}
\subsection{Requisiti non funzionali}
\subsubsection{App mobile}
\begin{itemize}
    \item Nella pagina dei dettagli di un albero collezionato deve essere mostrato anche il progetto associato.
    \item La pagina per il caricamento dei progressi deve fornire una breve guida e indicare le informazioni che verranno condivise.
    \item Il caricamento dei progressi non deve impiegare più 8 secondi altrimenti deve essere interrotto. A fine processo, anche in caso di errore, deve essere mostrata una schermata che informi l'utente dell'avvenuto caricamento dei dati.
\end{itemize}
\subsubsection{Totem}
\begin{itemize}
    \item L'utente deve poter depositare i progressi raggiunti in app in qualsiasi momento indipendentemente dalla schermata del totem. Deve quindi essere sempre accessibile il codice QR che identifica il totem in modo da poterlo scansionare tramite app.
    \item Il sistema deve essere scalabile in modo da permettere l'aggiunta di nuovi totem interattivi e gestire simultaneamente più richieste di condivisione dei dati dall'applicazione.
    \item L'interfaccia deve apparire familiare rispetto all'app mobile in modo da permettere un utilizzo immediato e non scoraggiare gli utenti all'interazione. Devono essere necessari un paio di tocchi per navigare tra le schermate o mostrare il codice QR.
    \item La sincronizzazione dei dati deve essere, per quanto possibile, immediata in modo da mostrare sempre dati aggiornati in tempo reale.
\end{itemize}

\subsection{Modello del dominio}
I dati del totem, che vengono memorizzati all'interno del database nel cloud, riguardano le informazioni sul totem stesso (ad esempio il nome e la sede UniBo in cui si trova) e tutti i progressi caricati tramite app dagli utenti.
Si ha quindi un'entità che rappresenta i progressi caricati dall'utente (\texttt{SharedData}) e un'entità che rappresenta il totem (\texttt{Totem}).
\begin{figure}[h!]
    \centering
    \includegraphics[width=0.6\textwidth]{img/totem/totemDomain.png}
    \caption[Modello del dominio del totem]{Modello del dominio del totem: in grassetto sono evidenziate le chiavi delle entità}
    \label{fig:totemDomain}
\end{figure}
%
%
%
%
\section{Design}
\subsection{Architettura e pattern}
Il sistema si compone di tre componenti principali: l'app mobile, il totem e il cloud database. I progressi dell'utente dall'app vengono caricati sul database quindi scaricati nel totem. In figura \ref{fig:communication-schema} viene mostrato lo schema delle interazioni fra i tre componenti del sistema durante la condivisione dei progressi.
Il database fornisce inoltre le informazioni sugli alberi e i relativi progetti sia al totem che all'app mobile.
\begin{figure} [h]
    \centering
    \includegraphics[width=0.7\textwidth]{img/arch-totem-app-dati.png}
    \caption{Schema della architettura e interazione fra app e totem}
    \label{fig:communication-schema}
\end{figure}

% Singleton pattern to manage the instance of a database connection.

Per quanto riguarda l'architettura interna del software, sia per il totem che per l'app mobile, viene utilizzata una forma del pattern architetturale MVI (\textit{Model-View-Intent}). La \textit{View} viene aggiornata dal DataManager (\textit{Model}) attraverso un flusso unidirezionale contenente lo stato della richiesta dell'utente (\textit{Intent}) che mantiene inoltre i dati del modello aggiornati da visualizzare.
Buona parte degli eventi scatenati dall'utente nella \textit{View} innescano un \textit{intent}.
Per esempio nell'app, dopo la scansione del codice qr di un albero viene avviato un \textit{intent} che va ad aggiornare il dominio; la \textit{View} nel frattempo viene costantemente informata sullo stato dell'operazione attraverso un flusso dati permettendo la reattività dell'interfaccia e la visualizzazione dei dati aggiornati.
\begin{figure}
    \centering
    \includegraphics[width=0.6\textwidth]{img/totem/mvi-schema.png}
    \caption{Diagramma delle classi per il pattern MVI}
    \label{fig:mviPattern}
\end{figure}

Per l'accesso ai dati è stato utilizzato il pattern \textit{Repository} che prevede una unica interfaccia definita dalla classe \texttt{DataManager} che fornisce un unico punto di accesso ai dati incapsulando l'elaborazione dei dati prelevati da provider diversi.
\texttt{FirebaseProvider} è l'unico provider di dati del totem mentre l'app ha anche \texttt{DatabaseProvider} e \texttt{PreferenciesProvider}.
Il repository comunica con i provider attraverso un'interfaccia DAO (\textit{Data Access Object}) specifica per ciascuno di questi e permette di definire le operazioni eseguibili separando l'aspetto implementativo.

\begin{figure}
    \centering
    \includegraphics[width=0.8\textwidth]{img/app/repository_pattern_scheme.png}
    \caption{Schema UML pattern Repository nell'app: l'interfaccia \texttt{DataManager} è il repository, mentre \texttt{DatabaseProvider}, \texttt{FirebaseProvider} e \texttt{PreferenciesProvider} sono le interfacce dei singoli fornitori e sorgenti di dati.}
    \label{fig:repoPattern}
\end{figure}

\subsection{Mockup}
Attraverso i \textit{mockup}, vengono presentate e spiegate le diverse schermate del totem.
Il termine \textit{mockup} o \textit{mock-up}, in italiano \enquote*{modello}, è una realizzazione a scopo illustrativo o espositivo di un oggetto o sistema, senza le complete funzioni dell'originale. Viene tipicamente sviluppato per fornire un'idea visiva del prodotto finale ai clienti, permettendo di effettuare correzioni o modifiche sulla bozza del prodotto ancora prima che si passi alla fase di sviluppo.

Per la creazione dei mockup è stato utilizzato il software Figma anche se in commercio ne esistono molti altri come ad esempio Adobe XD o Balsamiq.

\subsubsection{Mobile app}
La pagina per la condivisione dei dati, mostrata in figura \ref{fig:sharePage}, fornisce all'utente una breve spiegazione e informa quali dati verranno condivisi e mostrati pubblicamente nel totem.
Cliccando sul pulsante \enquote*{Procedi} si passa all'identificazione del totem tramite codice QR (figura \ref{fig:scanTotem}) e successivamente alla schermata di caricamento dei dati sul totem (figura \ref{fig:uploadinData}).
\begin{figure}[h]
    \centering
    \subfloat[Pagina di condivisone progressi]{
        \includegraphics[width=0.3\textwidth]{img/app/uploadPage.png}
        \label{fig:sharePage}
    }
    \subfloat[Scansione QR code totem]{
        \includegraphics[width=0.3\textwidth]{img/app/uploadProgress.png}
        \label{fig:scanTotem}
    }
    \subfloat[Schermata di caricamento progressi]{
        \includegraphics[width=0.3\textwidth]{img/app/uploadingPage.png}
        \label{fig:uploadinData}
    }
    \caption{}
    \label{fig:shareDataApp}
\end{figure}

\subsubsection{Totem}
La struttura dell'interfaccia del totem si suddivide in due sezioni principali: una colonna a sinistra contenente la barra di navigazione e la sezione con il QR code del totem e lo spazio restante a destra dove viene visualizzata la pagina selezionata (figura \ref{fig:viewStruct}).
\begin{figure}
    \centering
    \includegraphics[width=0.8\textwidth]{img/totem/mainStructure.png}
    \caption{Layout generale della UI del totem composto da tre blocchi: barra di navigazione (in verde), sezione codice QR (in arancione) e schermata selezionata (in grigio)}
    \label{fig:viewStruct}
\end{figure}

Vengono mantenuti i colori, il particolare dell'erba (figura \ref{fig:grassDetails}) e altre grafiche e icone, in modo da avere coerenza e continuità fra gli applicativi.
\begin{figure}
    \centering
    \includegraphics[width=0.4\textwidth]{img/totem/grassDetail.png}
    \caption{Dettaglio dell'erba presente nella'app e nel totem}
    \label{fig:grassDetails}
\end{figure}

In figura \ref{fig:depositIconsQR} viene mostrata l'icona del pulsante che mostra il codice QR del totem. Al tocco del pulsante, questo si ingrandisce rendendo visibile al suo interno il codice QR (figura \ref{fig:showQR}).
\begin{figure}
    \centering
    \subfloat[Codice QR del totem nascosto]{
        \includegraphics[width=3cm]{img/totem/depositIcon.png}
        \label{fig:hideQR}
    }
    \vspace{1cm}
    \subfloat[Codice QR del totem visibile]{
        \includegraphics[width=3cm]{img/totem/depositQR.png}
        \label{fig:showQR}
    }
    \caption[Icone del QR code nel totem]{Icone del pulsante per mostrare il codice QR del totem e avviare la condivisione}
    \label{fig:depositIconsQR}
\end{figure}

Per quanto riguarda la homepage sono stati fatti due mockup con design differenti: nel primo vengono mostrati gli utenti come piccoli alberi disposti su delle mensole (figura \ref{fig:shelfHome}), nel secondo viene utilizzato un layout a griglia e gli utenti vengono rappresentati come chiome di alberi visti dall'alto (figura \ref{fig:forestHome}) ricreando un vero e proprio boschetto virtuale. In entrambi i casi il tocco di un elemento espande il pannello dei dettagli utente e le chiome degli alberi variano in base al livello raggiunto.
\begin{figure}
    \centering
    \subfloat[Visualizzazione a mensole]{
        \includegraphics[width=0.45\textwidth]{img/totem/shelfViewHome.png}
        \label{fig:shelfHome}
    }
    \subfloat[Visualizzazione a bosco]{
        \includegraphics[width=0.45\textwidth]{img/totem/forestViewHome.png}
        \label{fig:forestHome}
    }
    \caption[Alternative di homepage nel totem]{Design alternativi per la homepage del totem}
    \label{fig:homepages}
\end{figure}

Il pannello dei dettagli utente si trova in alto a destra nella homepage e mostra il nickname, i progressi e il livello raggiunti dell'utente selezionato. Sono stati pensati tre diversi design  che differiscono per decorazione; si ha quindi la mensola, il cordino e lo stile tondeggiante (rispettivamente figura \ref{fig:shelfDetail}, \ref{fig:cordDetail}, \ref{fig:roundDetail}).
\begin{figure}
    \centering
    \subfloat[Stile mensola]{
        \includegraphics[width=0.30\textwidth]{img/totem/shelfDetail.png}
        \label{fig:shelfDetail}
    }
    \subfloat[Stile tendina]{
        \includegraphics[width=0.30\textwidth]{img/totem/cordDetail.png}
        \label{fig:cordDetail}
    }
    \subfloat[Stile curvo]{
        \includegraphics[width=0.28\textwidth]{img/totem/roundDetail.png}
        \label{fig:roundDetail}
    }
    \caption{Tre diversi stili per il pannello dei dettagli nel totem}
    \label{fig:detailBanner}
\end{figure}

La pagina delle statistiche (figura \ref{fig:statsPage}) mostra sei contatori (figura \ref{fig:statCircle}) ciascuno per un dato disposti in una griglia di tre colonne e due righe. Il contatore è stato pensato come un cerchio colorato con all'interno un'icona rappresentativa della statistica. Al tocco del singolo elemento viene mostrata una breve descrizione del dato.
\begin{figure}
    \centering
    \subfloat[]{
        \includegraphics[width=0.3\textwidth]{img/totem/plantedTreeStats.png}
        \label{fig:iconStats}
    }
    \subfloat[]{
        \includegraphics[width=0.3\textwidth]{img/totem/plantedTreeStatsdescr.png}
        \label{fig:descrShowedStats}
    }
    \caption[Pagina delle statistiche nel totem]{Singolo contatore della schermata delle statistiche nel totem. Al tocco del contatore viene mostrata o nascosta la descrizione passando da \ref{fig:iconStats} a \ref{fig:descrShowedStats} e viceversa}
    \label{fig:statCircle}
\end{figure}

\begin{figure}
    \centering
    \includegraphics[width=0.8\textwidth]{img/totem/statsPage.png}
    \caption[Pagina delle statistiche nel totem]{Pagina delle statistiche nel totem che presenta sei contatori disposti a griglia}
    \label{fig:statsPage}
\end{figure}

\begin{figure}
    \centering
    \includegraphics[width=0.8\textwidth]{img/totem/topchartPage.png}
    \caption[Classifica Top10 nel totem]{Pagina della classifica dei 10 migliori utenti}
    \label{fig:chartPage}
\end{figure}

\begin{figure}
    \centering
    \includegraphics[width=0.8\textwidth]{img/totem/infoPage.png}
    \caption[Pagina delle informazioni nel totem]{Pagina delle informazioni nel totem che prevede un design semplice con delle \textit{tiles} (\enquote*{piastrelle}) disposte a griglia. Al tocco di una tile si apre un pop-up che mostra maggiori informazioni.}
    \label{fig:infoPage}
\end{figure}
%
%
%

\section{Tecnologie impiegate}
In questo capitolo vengono presentate le tecnologie utilizzate per lo sviluppo della applicativo del totem e della applicazione mobile. Per entrambi i dispositivi si è sviluppato utilizzando il framework Flutter, per l'integrazione e gestione dei dati è stato utilizzato il servizio cloud Firebase Realtime mentre per l'identificazione degli alberi e del totem sono stati utilizzati i codici QR.

\subsection{Flutter}
Flutter \cite{flutter} è un kit di sviluppo software (\textit{Software Development Kit}, SDK) \textit{open source} che comprende diversi strumenti (\textit{DevTools}) per lo sviluppo di applicazioni multi-piattaforma con particolare attenzione alla interfaccia utente (UI, \textit{User Interface}). All'interno del SDK è contenuto il framework Flutter che utilizza il linguaggio di programmazione Dart, anch'esso sviluppato da Google, e librerie grafiche predefinite per le piattaforme Android e iOS.

Il linguaggio Dart è un linguaggio di programmazione ad alto livello multi paradigma, influenzato da altri linguaggi come il C e Java, ed è ottimizzato per lo sviluppo veloce di applicazioni su qualsiasi piattaforma nativa e web.

\subsubsection{Plugin utilizzati}
Di seguito si elencano i plugin flutter utilizzati nello sviluppo dell'applicativo del totem:

\begin{itemize}
    \item \texttt{flutter\_svg} \cite{providerPlugin}, per la visualizzazione delle immagini vettoriali utilizzate per la rappresentazione dei livelli utente e alcuni elementi grafici dell'interfaccia;
    \item \texttt{firebase\_core} \cite{firebaseCorePlugin} e \texttt{firebase\_database} \cite{firebaseDatabasePlugin}, pacchetti per poter utilizzare i servizi Firebase e quindi l'accesso al database;
    \item \texttt{barcode} \cite{barcodePlugin}, è una libreria per la generazione di codici a barre e QR;
    \item \texttt{provider} \cite{providerPlugin}, permette di utilizzare coma maggiore facilità e semplicità il widget flutter InheritedWidget che serve a propagare una informazione nell'albero dei widget.
    
\end{itemize}
\subsection{Firebase Realtime}
Firebase Realtime \cite{firebase} è un database non relazionale (NoSQL) memorizzato nel cloud. In quanto NoSQL, i dati e le relazioni non vengono memorizzati in tabelle ma in documenti, archivi \textit{wide-column} o grafi (nodi e archi) a seconda della tipologia del database.
Firebase Realtime è di tipo documentale e quindi fa uso di documenti in formato JSON (\textit{JavaScript Object Notation}) per salvare i dati. Questo genere di memorizzazione permette maggiore flessibilità dello schema permettendo l'aggiunta o la modifica del modello dei dati (es. aggiunta o modifica di un campo). Inoltre i database NoSQL consentono una maggiore velocità e agilità di memorizzazione ed elaborazione oltre che offrire una maggiore scalabilità.
%
\subsection{Codici QR}
Il codice QR (\textit{Quick Response}) è un tipo di codice a barre a matrice (codice 2D) sviluppato e presentato nel 1994 dalla compagnia giapponese Denso Wave che si compone di tanti piccoli quadrati neri e bianchi, chiamati moduli, disposti in una matrice \cite{qrCodeDensoWave}.
I codici QR permettono di memorizzare una quantità variabile di dati che dipende dalla tipologia di dato, dalla versione utilizzata e dal livello di correzione dell'errore. La dimensione della matrice è definita dalla versione utilizzata e anche dalla tipologia di codice QR (Modello 1 o 2, \textit{Micro QR Code}, \textit{rMQR Code}, \textit{SQRC}, Frame QR).

La versione più comune di \textit{QR code} è di tipo \textit{Model 1 o 2} di livello 3, presenta 29x29 moduli ed è in grado di memorizzare fino ad un massimo di 77 caratteri alfanumerici con la correzione dell'errore del 7\% circa.

\section{Strumenti di sviluppo}
\subsection{Git}
Git \cite{gitSite} è un \textit{Version Control System} VCS cioè un sistema di controllo delle versioni dei file. I cambiamenti di uno o più file nel tempo vengono memorizzati e questo permette di mantenere una cronologia dei cambiamenti e di ripristinare una versione precedente di file, cartelle o interi progetti. Esistono diverse tipologie di VCS: locale, centralizzato o distribuito.
Rispetto ai tradizionali VCS, Git non memorizza le differenze dei file (\textit{delta-based version control}) ma memorizza una istantanea di come appaiono i file nel \textit{filesystem} nell'istante in cui si effettua un salvataggio di versione (\textit{commit}). Git considera i suoi dati più come a un flusso d'istantanee del progetto che viene tracciato. Git possiede un sistema di branching/merging avanzati e di gestione dei conflitti.

\subsection{Visual Studio Code}
Visual Studio Code \cite{vsCodeSite} è un editor di codice sorgente, sviluppato da Microsoft per Windows, Mac e Linux. Prevede diverse funzionalità utili per la scrittura di codice come la previsione e suggerimento di codice (IntelliSense) e gli strumenti di segnalazione e correzione degli errori, che i comuni editor di testo non possiedono.
Può essere utilizzato con la maggior parte dei linguaggi di programmazione (es. C, Java, PHP) ed è possibile aggiungerne di nuovi ed estendere le sue capacità attraverso dei \textit{plugin} scaricabili direttamente all'interno di Visual Studio Code.
Viste le qualità di questo editor e al suo utilizzo pregresso si è deciso di utilizzarlo per lo sviluppo di questo progetto con l'integrazione del \textit{plugin} ufficiale di Flutter che aggiunge ulteriori strumenti di analisi e \textit{debug}.